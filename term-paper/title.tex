% !TEX root = documentation.tex


\titlehead{BSc Computational and Data Science\\CDS205 Computer Science\\Dozent: Prof. Dr. Ana Petrus\hfill}
\title{Analyse von nachträglichen Artikeländerungen bei Schweizer online Medien}
\subtitle{}
\author[1,*]{Dario Hollbach}
\affil[1]{Fachhochschule Graubünden}
\affil[*]{E-Mail Adresse: dario.hollbach@stud.fhgr.com}
\date{\today}
\maketitle
\thispagestyle{empty}

\begin{abstract}
Im Gegensatz zu Printmedien können Artikel in online Medien auch nach der Veröffentlichung modifiziert werden. 
Diese Arbeit untersucht die Häufigkeit und Art solcher nachträglichen Änderungen bei den Schweizer online Medien 20 Minuten, Watson und SRF News. 
Hierfür wurde ein System entwickelt, das über einen Zeitraum von zehn Tagen systematisch Artikel erfasst und Versionen vergleicht. 
Die identifizierten Änderungen wurden mithilfe einer Web-Applikation visualisiert und manuell als Tippfehler oder inhaltliche Anpassung klassifiziert. 
Die Analyse zeigt, dass rund jeder fünfte Artikel mindestens eine Änderung erfährt. 
Die Ergebnisse geben Aufschluss über die Korrekturprozesse und die Dynamik des digitalen Journalismus in der Schweiz.
\end{abstract}