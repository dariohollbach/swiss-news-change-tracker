% !TEX root = documentation.tex


\titlehead{BSc Computational and Data Science\\CDS205 Computer Science\\Dozentin: Prof. Dr. Ana Petrus\hfill}
\title{Analyse von nachträglichen Artikeländerungen bei Schweizer Online-Zeitschriften}
\subtitle{}
\author[1,*]{Dario Hollbach}
\affil[1]{Fachhochschule Graubünden}
\affil[*]{E-Mail Adresse: dario.hollbach@stud.fhgr.ch}
\date{\today}
\maketitle
\thispagestyle{empty}

\begin{abstract}
Im Gegensatz zu Printmedien können Artikel in Online-Zeitschriften auch nach der Veröffentlichung modifiziert werden. 
Diese Arbeit untersucht die Häufigkeit und Art solcher nachträglichen Artikeländerungen bei den Schweizer Zeitschriften \textit{20 Minuten}, \textit{Watson} und \textit{SRF News}. 
Hierfür wurden eine Python-Applikation entwickelt, welche auf einem Raspberry-Pi über den Zeitraum von zehn Tagen systematisch Artikel erfasst und Versionen vergleicht. 
Die in der Datenbank gespeicherten Änderungen wurden mithilfe einer Web-Applikation visualisiert und manuell als Tippfehler oder inhaltliche Anpassung klassifiziert. 
Die Analyse zeigt, dass rund jeder fünfte Artikel mindestens ein Mal geändert wurde. 
Die Ergebnisse zeigen, wie Schweizer Online-Zeitschriften Korrekturen und Anpassungen vornehmen.
\end{abstract}